% --------------------------------------------------------
% DEFINIÇÕES DO DOCUMENTO
% --------------------------------------------------------

\documentclass[
	% -- opções da classe memoir --
	12pt,				% tamanho da fonte
	openright,			% capítulos começam em pág ímpar (insere página vazia caso preciso)
	oneside,			% para impressão em verso e anverso. Oposto a twoside
	a4paper,			% tamanho do papel.
	% -- opções da classe abntex2 --
	%chapter=TITLE,		% títulos de capítulos convertidos em letras maiúsculas
	%section=TITLE,		% títulos de seções convertidos em letras maiúsculas
	%subsection=TITLE,	% títulos de subseções convertidos em letras maiúsculas
	%subsubsection=TITLE,% títulos de subsubseções convertidos em letras maiúsculas
	% -- opções do pacote babel --
	english,			% idioma adicional para hifenização
	french,				% idioma adicional para hifenização
	spanish,			% idioma adicional para hifenização
	brazil,				% o último idioma é o principal do documento
]{lib/abntex2}


% --------------------------------------------------------
% PACOTES
% --------------------------------------------------------

\usepackage{cmap}				% Mapear caracteres especiais no PDF
\usepackage{lmodern}			% Usa a fonte Latin Modern
\usepackage[T1]{fontenc}		% Selecao de codigos de fonte.
\usepackage[utf8]{inputenc}		% Codificacao do documento (conversão automática dos acentos)
\usepackage{lastpage}			% Usado pela Ficha catalográfica
\usepackage{indentfirst}		% Indenta o primeiro parágrafo de cada seção.
\usepackage{color}				% Controle das cores
\usepackage{graphicx}			% Inclusão de gráficos
\usepackage{lipsum}				% para geração de du

\let\printglossary\relax
\let\theglossary\relax
\let\endtheglossary\relax
\usepackage{lib/update-abntex}

\usepackage[brazilian,hyperpageref]{}	 % Paginas com as citações na bibl
\usepackage{microtype} 

\usepackage{silence}
%Disable all warnings issued by latex starting with "You have..."
\WarningFilter{latex}{You have requested package}
%\usepackage[alf, abnt-etal-text=emph]{lib/abntex2cite}	% Citações padrão ABNT

\RequirePackage[
   alf,
   abnt-repeated-title-omit = yes,
   abnt-emphasize = bf,
   abnt-etal-list=0,
   abnt-etal-text=emph,
   abnt-etal-cite=2,
 ]{abntex2cite}

\usepackage[br]{lib/nicealgo}       % Pacote para criação de algoritmos
\usepackage{lib/customizacoes}      % Pacote de customizações do abntex2

\usepackage{listings}
\usepackage[normalem]{ulem} % Strikethrough package

% --------------------------------------------------------
% CONFIGURAÇÕES DE PACOTES
% --------------------------------------------------------

% Configurações do pacote listing
\renewcommand{\lstlistingname}{Código} %Mudança no caption do listing para Código
\renewcommand{\lstlistlistingname}{Lista de códigos} %Mudança no caption da lista de listings.

% Contagem de códigos sem incluir o número do capítulo
\usepackage{chngcntr}
\AtBeginDocument{\counterwithout{lstlisting}{chapter}}

% Configurações do pacote backref
\renewcommand{\familydefault}{\sfdefault}
% Usado sem a opção hyperpageref de backref
% \renewcommand{\backrefpagesname}{Citado na(s) página(s):~}
% Texto padrão antes do número das páginas
% \renewcommand{\backref}{}
% Define os textos da citação
% \renewcommand*{\backrefalt}[4]{
% 	\ifcase #1 %
% 		Nenhuma citação no texto.%
% 	\or
% 		Citado na página #2.%
% 	\else
% 		Citado #1 vezes nas páginas #2.%
% 	\fi}%


% ---
% Posiciona figuras e tabelas no topo da página quando adicionadas sozinhas
% em um página em branco. Ver https://github.com/abntex/abntex2/issues/170
\makeatletter
\setlength{\@fptop}{5pt} % Set distance from top of page to first float
\makeatother
% ---

% ---
% Possibilita criação de Quadros e Lista de quadros.
% Ver https://github.com/abntex/abntex2/issues/176
%
\newcommand{\quadroname}{Quadro}
\newcommand{\listofquadrosname}{Lista de quadros}

\newfloat[chapter]{quadro}{loq}{\quadroname}
\newlistof{listofquadros}{loq}{\listofquadrosname}
\newlistentry{quadro}{loq}{0}

% configurações para atender às regras da ABNT
\setfloatadjustment{quadro}{\centering}
\counterwithout{quadro}{chapter}
\renewcommand{\cftquadroname}{\quadroname\space} 
\renewcommand*{\cftquadroaftersnum}{\hfill--\hfill}

\setfloatlocations{quadro}{hbtp}
% ---

% --------------------------------------------------------
% INFORMAÇÕES DE DADOS PARA CAPA E FOLHA DE ROSTO
% --------------------------------------------------------

\titulo{Explorando Algoritmos de Compressão de Dados: Teoria, Implementação e Desempenho}
\autor{Gustavo Yujii Silva Kadooka}
\newcommand{\RA}{221021582}
\local{Bauru}
\data{Março/2025}
\orientador{Andréa Carla Gonçalves Vianna}
\instituicao{
  Universidade Estadual Paulista ``Júlio de Mesquita Filho''
  \par
  Faculdade de Ciências
  \par
  Ciência da Computação}
\tipotrabalho{Proposta para Trabalho de Conclusão de Curso}
\preambulo{Proposta para Trabalho de Conclusão de Curso do Curso de Bacharelado em Ciência da Computação da Universidade Estadual Paulista ``Júlio de Mesquita Filho'', Faculdade de Ciências, Campus Bauru.}


% --------------------------------------------------------
% CONFIGURAÇÕES PARA O PDF FINAL
% --------------------------------------------------------

% alterando o aspecto da cor azul
\definecolor{blue}{RGB}{41,5,195}

% informações do PDF
\makeatletter
\hypersetup{
  %pagebackref=true,
  pdftitle={\@title},
  pdfauthor={\@author},
  pdfsubject={\imprimirpreambulo},
  pdfcreator={LaTeX with abnTeX2},
  pdfkeywords={abnt}{latex}{abntex}{abntex2}{trabalho acadêmico},
  colorlinks=true,    % false: boxed links; true: colored links
  linkcolor=black,    % color of internal links
  citecolor=black,    % color of links to bibliography
  filecolor=magenta,  % color of file links
  urlcolor=black,
  bookmarksdepth=4
}
\makeatother


% --------------------------------------------------------
% ESPAÇAMENTOS ENTRE LINHAS E PARÁGRAFOS
% --------------------------------------------------------

% O tamanho do parágrafo é dado por:
\setlength{\parindent}{1.3cm}

% Controle do espaçamento entre um parágrafo e outro:
\setlength{\parskip}{0.2cm}


% --------------------------------------------------------
% COMPILANDO O ÍNDICE
% --------------------------------------------------------

\makeindex

% ---
% GLOSSARIO
% ---
%\makeglossaries
% ---
% Exemplo de configurações do glossairo
\renewcommand*{\glsseeformat}[3][\seename]{\textit{#1}  
 \glsseelist{#2}}
% ---


% --------------------------------------------------------
% INÍCIO DO DOCUMENTO
% --------------------------------------------------------

\begin{document}

% Seleciona o idioma do documento (conforme pacotes do babel)
\selectlanguage{brazil}

% Retira espaço extra obsoleto entre as frases.
\frenchspacing


% --------------------------------------------------------
% ELEMENTOS PRÉ-TEXTUAIS
% --------------------------------------------------------

% Capa
\imprimircapaproposta

% Folha de rosto
% (o * indica que haverá a ficha bibliográfica)
\imprimirfolhaderosto



% --------------------------------------------------------
% SUMÁRIO
% --------------------------------------------------------

% inserir o sumario
\pdfbookmark[0]{\contentsname}{toc}
\tableofcontents*
\cleardoublepage


% --------------------------------------------------------
% ELEMENTOS TEXTUAIS
% --------------------------------------------------------

\pagestyle{simple}

% Arquivos .tex do texto, podendo ser escritos em um único arquivo ou divididos da forma desejada
\chapter{Introdução}
\label{c.introducao}

A compressão de dados é uma técnica essencial na ciência da computação, utilizada para reduzir o tamanho dos arquivos, otimizando o uso de recursos e acelerando a transmissão de dados, o que se torna crucial na era digital em que vivemos~\cite{salomon2007data}.  

O desenvolvimento dessa área teve início na década de 1950, com os primeiros métodos que visavam otimizar a utilização do espaço de armazenamento. Entre esses métodos, destaca-se o algoritmo de Huffman, que utiliza uma técnica de codificação de prefixo variável. Nesse algoritmo, símbolos mais frequentes recebem códigos de comprimento menor, enquanto símbolos menos frequentes recebem códigos mais longos~\cite{salomon2007data}. O método de Huffman é amplamente empregado em compressão sem perdas e se tornou fundamental para a criação de arquivos compactados em formatos como ZIP e GZIP~\cite{deutsch1996gzip}. Sua simplicidade, aliada à sua eficácia, fez com que se consolidasse como um dos pilares da compressão de dados.  

Nos anos seguintes, com o aumento da demanda por transmissão de grandes volumes de dados surgiram novos algoritmos. Um exemplo significativo é o LZ77 (Lempel-Ziv 77), introduzido por Abraham Lempel e Jacob Ziv em 1977~\cite{ziv1977universal}. O LZ77 aplica uma técnica de compressão baseada em dicionários, em que sequências de dados repetidas são substituídas por referências a posições anteriores. Essa abordagem inspirou a criação de outros algoritmos, como o LZW (Lempel-Ziv-Welch), que aprimora o LZ77 ao criar dinamicamente um dicionário de \textit{strings} enquanto os dados são comprimidos~\cite{welch1984technique}. O LZW é particularmente famoso pela sua aplicação no formato de compressão GIF, bem como em arquivos TIFF.  

Além desses, o GZIP se destaca como outro algoritmo amplamente utilizado, especialmente em sistemas Unix e na web. O GZIP combina a codificação de Huffman com o algoritmo LZ77, permitindo uma compressão eficiente e rápida, sem perda de dados~\cite{deutsch1996gzip}. Sua popularidade decorre da combinação de alta taxa de compressão e facilidade de descompressão, sendo uma escolha recorrente em aplicações como transferência de arquivos e armazenamento de dados.  

Os métodos de Huffman, LZ77, LZW e GZIP, formam a espinha dorsal dos algoritmos clássicos de compressão de dados e continuam sendo amplamente utilizados em diversas áreas, apesar do surgimento de novas abordagens.  

Atualmente, a compressão de dados continua a ser um campo dinâmico e em constante evolução, com diversos algoritmos sendo constantemente estudados e aprimorados. Entre os mais recentes, destacam-se os métodos baseados em compressão adaptativa, que ajustam seus parâmetros conforme as características dos dados a serem comprimidos, oferecendo maior eficiência dependendo do conteúdo~\cite{salomon2007data}. Um exemplo é o Brotli, um algoritmo de compressão sem perdas desenvolvido pela Google, amplamente utilizado para comprimir arquivos em navegadores web. O Brotli combina codificação de Huffman e transformações de fluxo de dados, permitindo taxas de compressão superiores às oferecidas por algoritmos tradicionais como o GZIP~\cite{alakuijala2016brotli}.  

Além disso, algoritmos de compressão em tempo real têm ganhado relevância devido ao crescente volume de dados gerados em \textit{streaming} de vídeo e nas comunicações móveis. O Zstandard (ou Zstd), desenvolvido pelo Facebook, é um exemplo notável. Ele oferece uma excelente combinação entre alta taxa de compressão e velocidade, podendo ser utilizado tanto em compressões em tempo real quanto em grandes volumes de dados~\cite{collet2016zstandard}.  

Em áreas como compressão de imagens e áudio, técnicas baseadas em aprendizado de máquina e redes neurais também têm atraído atenção. No campo da compressão de imagens, os Modelos Generativos Adversariais (GANs) estão sendo explorados para desenvolver algoritmos que geram representações comprimidas de alta qualidade, mantendo os detalhes visuais. Na compressão de áudio, algoritmos baseados em redes neurais, como o WaveNet~\cite{wavenet}, estão sendo experimentados para aprimorar a qualidade da compressão, tanto com perdas quanto sem perdas.  

Outro campo de grande importância é a compressão de vídeo, especialmente para plataformas de streaming como Netflix e YouTube. Algoritmos como o HEVC (\textit{High Efficiency Video Coding}), uma evolução do H.264, continuam a ser aprimorados para garantir uma compressão mais eficiente, sem perda significativa de qualidade. Além disso, o desenvolvimento de novos codecs como o VVC (\textit{Versatile Video Coding}) está em andamento, com a promessa de oferecer compressão ainda mais eficiente para vídeos de alta definição e 4K.  

A análise de desempenho, que inclui a eficiência em termos de tempo de execução e taxa de compressão, também desempenha um papel crucial. Com o aumento exponencial da quantidade de dados gerados e transmitidos na sociedade moderna, os algoritmos de compressão tornaram-se ainda mais essenciais. A eficiência na compressão não só reduz os custos de armazenamento e acelera a transmissão de dados, mas também impacta diretamente áreas como \textit{streaming} de vídeos, comunicação móvel, redes de computadores e sistemas de armazenamento em nuvem. Compreender os diferentes métodos e avaliar sua aplicabilidade em cenários diversos é fundamental para o desenvolvimento de soluções mais eficazes e eficientes~\cite{salomon2007data}.

Desta forma, este trabalho propõe a exploração dos algoritmos clássicos de compressão de dados, abordando seus aspectos teóricos, implementação prática e análise de desempenho.

\chapter{Problema}
\label{c.problema}

asd
\chapter{Justificativa}
\label{c.justificativa}
\chapter{Objetivos}
\label{c.objetivos}

\section{Objtivo Geral}

\section{Objtivos Específicos}
\chapter{Metodologia}
\label{c.metodologia}

O estudo tem caráter \textbf{exploratório e descritivo}, pois busca compreender e comparar o desempenho de algoritmos clássicos de compressão de dados em diferentes tipos de arquivos. Além disso, trata-se de uma pesquisa \textbf{experimental e bibliográfica}, pois envolve tanto a revisão da literatura sobre os algoritmos estudados quanto a implementação prática e execução de testes para coleta de dados.

\section{Delineamento da pesquisa}

A pesquisa será conduzida em duas etapas principais:

\begin{itemize}
    \item \textbf{Levantamento de literatura}: estudo de referências acadêmicas sobre os algoritmos Huffman, LZ77, LZW e GZIP, incluindo seus princípios teóricos, implementação e aplicações.
    \item \textbf{Análise experimental}: implementação dos algoritmos e realização de testes comparativos, medindo a taxa de compressão e o tempo de execução em de arquivos de texto, áudio e imagem.
\end{itemize}

Com esse delineamento metodológico, espera-se obter uma avaliação quantitativa e qualitativa dos algoritmos estudados, possibilitando uma conclusão fundamentada sobre suas aplicações ideais.

\section{Levantamento de literatura}
O material bibliográfico será composto por livros, artigos científicos, teses, dissertações e fontes digitais relevantes para os algoritmos de compressão de dados. Serão consultados materiais teóricos que abordam os princípios, implementações e aplicações dos algoritmos Huffman, LZ77, LZW e GZIP. Além disso, serão utilizadas fontes que detalham a análise e comparação de desempenho desses algoritmos.

\section{Análise experimental}
Os algoritmos serão implementados nas linguagens de programação \texttt{C} ou \texttt{C++}. 

\subsection{Coleta de dados}
Os dados serão coletados por meio da execução dos algoritmos em um ambiente controlado, utilizando conjuntos de arquivos representativos dos três principais tipos de dados abordados: 

\begin{itemize}
    \item \textbf{Texto}: arquivos de livros e artigos científicos em formato \texttt{.txt}.
    \item \textbf{Imagem}: arquivos \texttt{.bmp}, que utilizam compressão sem perdas.
    \item \textbf{Áudio}: pequenos arquivos de áudio sem perdas em formato \texttt{.wav}.
\end{itemize}

Cada algoritmo será executado sobre os mesmos conjuntos de dados e os seguintes parâmetros serão registrados:

\begin{itemize}
    \item Taxa de compressão (\% de redução do tamanho do arquivo);
    \item Tempo de execução (medido em milissegundos);
\end{itemize}

\subsection{Análise dos dados}

Os resultados serão tabulados e analisados estatisticamente para identificar padrões e diferenças entre os algoritmos. A análise será feita considerando:

\begin{itemize}
    \item Comparação direta entre a taxa de compressão e tempo de execução;
    \item Eficiência relativa dos algoritmos para cada tipo de dado;
    \item Gráficos de desempenho desenvolvidos utilizando a linguagem de programação \texttt{Python} com as bibliotecas \texttt{pandas} e \texttt{matplotlib} a fim de facilitar a interpretação. Para gerar os gráficos coletaremos os dados e os colocaremos em um arquivo \texttt{.csv}. 
\end{itemize}

\subsection*{Ambiente controlado}

Os experimentos serão conduzidos em um ambiente computacional padronizado para garantir reprodutibilidade. 

\begin{itemize}
    \item Sistema operacional: Windows 11 Pro com Debian WSL;
    \item Hardware: Processador	AMD Ryzen 3 PRO 8300GE w/ Radeon 740M Graphics, 3501 Mhz, 4 Núcleo(s), 8 Processador(es) Lógico(s), 16GB RAM.
\end{itemize}





\chapter{Cronograma}
\label{c.cronograma}

Os arquivos estão sendo concatenados. Podemos continuar a nossa escrita em outro arquivo .tex desde que ele seja importado no projeto principal, que é sempre o utilizado para efetuar a compilação.

% --------------------------------------------------------
% REFERÊNCIAS BIBLIOGRÁFICAS
% --------------------------------------------------------

\bibliography{chapters/referencias}


% --------------------------------------------------------
% ÍNDICE REMISSIVO
% --------------------------------------------------------

%\printindex


% --------------------------------------------------------
% FINAL DO DOCUMENTO
% --------------------------------------------------------

\end{document}
