\chapter{Cronograma}
\label{c.cronograma}

O cronograma da Tabela~\ref{tab:cronograma} apresenta a distribuição das atividades do Trabalho de Conclusão de Curso ao longo do período disponível, considerando os prazos estipulados pela instituição.

\begin{table}[htb]
    \centering
    \renewcommand{\arraystretch}{1.3}
    \caption{Cronograma do Projeto} % Título dentro da tabela
    \begin{tabular}{|c|c|c|c|c|c|c|c|c|}
        \hline
        \textbf{Etapas} & \textbf{ABR} & \textbf{MAIO} & \textbf{JUN} & \textbf{JUL} & \textbf{AGO} & \textbf{SET} & \textbf{OUT} & \textbf{NOV} \\
        \hline
        1 & X &   &   &   &   &   &   &   \\
        \hline
        2 & X & X & X & X &   &   &   &   \\
        \hline
        3 &   & X & X & X & X &   &   &   \\
        \hline
        4 &   &   & X & X & X &   &   &   \\
        \hline
	5 &   & X & X & X & X & X & X &   \\
	\hline
    \end{tabular}
    \vspace{2mm} \\ % Espaçamento entre a tabela e a fonte
    \textbf{Fonte:} elaborada pelo autor. % Fonte dentro da tabela
    \label{tab:cronograma} % Rótulo para referência no texto
\end{table}

\section*{Etapas}
\begin{enumerate}
    \item Levantamento bibliográfico: Pesquisa sobre os algoritmos de compressão. O material será composto por livros, artigos científicos, teses, dissertações e fontes digitais. 
    \item Desenvolvimento do \textit{back-end}: Implementação dos algoritmos de compressão em \texttt{C}/\texttt{C++}
    \item Desenvolvimento do \textit{front-end}: Implementação de uma interface gráfica usando a biblioteca \texttt{GTK} para selecionar arquivos e visualizar resultados.
    \item Testes e análise de desempenho: Testar os algoritmos, coletar dados de tempo de execução e percentagem de compressão, criando gráficos com a linguagem \texttt{Python} para visualização e análise.
    \item Escrita da monografia: Redação do conteúdo da monografia.
\end{enumerate}
