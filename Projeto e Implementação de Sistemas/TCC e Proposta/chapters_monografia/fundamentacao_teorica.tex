\chapter{Fundamentação teórica}
\label{c.fundamentacao_teorica}

Este capítulo apresenta os fundamentos teóricos que embasam o desenvolvimento deste trabalho. Para compreender o funcionamento e as diferenças entre os algoritmos de compressão de dados abordados, é necessário introduzir alguns conceitos essenciais.

Primeiramente, será apresentada a teoria da informação, a qual fornece o embasamento matemático para a compressão de dados sem perdas, por meio de conceitos como entropia e redundância. Em seguida, serão discutidos os princípios gerais da compressão de dados, com foco nas técnicas sem perdas, suas aplicações e métricas de desempenho. Por fim, serão detalhados os algoritmos clássicos de compressão sem perdas utilizados neste trabalho: Huffman, LZ77, LZW e GZIP, incluindo suas características, funcionamento e aplicações práticas.

A estrutura deste capítulo está organizada da seguinte forma:

\begin{itemize}
    \item \textbf{Seção 3.1} -- Apresenta os principais conceitos da Teoria da Informação, com ênfase na entropia de Shannon e seu papel na compressão de dados;
    \item \textbf{Seção 3.2} -- Discute os fundamentos da compressão de dados sem perdas, destacando seus objetivos, métricas e distinções em relação à compressão com perdas;
    \item \textbf{Seção 3.3} -- Descreve detalhadamente os algoritmos Huffman, LZ77, LZW e GZIP, explicando seus princípios de funcionamento, vantagens, limitações e casos de uso.
\end{itemize}

\section{Teoria da informação}

\section{Compressão de dados}

\section{Algoritmos clássicos de compressão sem perdas}

