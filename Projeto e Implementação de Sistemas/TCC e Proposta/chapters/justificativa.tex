\chapter{Justificativa}
\label{c.justificativa}

Este estudo justifica-se pela necessidade de compreender de forma aprofundada como diferentes algoritmos clássicos de compressão se comportam em termos de taxa de compressão e tempo de execução, analisando sua aplicabilidade em diferentes cenários \cite{salomon2007data}. Embora os algoritmos de Huffman, LZ77, LZW e GZIP sejam amplamente utilizados, muitas vezes a escolha entre eles é feita sem uma análise detalhada de suas vantagens e limitações para tipos específicos de dados \cite{ziv1977universal, welch1984technique}. Um estudo comparativo rigoroso pode fornecer \textit{insights} valiosos para desenvolvedores, engenheiros de software e administradores de sistemas que buscam otimizar o uso de recursos computacionais.

Além disso, a pesquisa contribui para o conhecimento técnico ao oferecer um panorama detalhado da eficiência desses algoritmos, permitindo um entendimento mais preciso sobre sua aplicabilidade em diferentes tipos de arquivos. Os resultados obtidos podem servir como referência para a escolha do algoritmo mais adequado em aplicações práticas, como armazenamento de dados compactados, otimização de tráfego em redes e sistemas embarcados com restrições de processamento e memória \cite{deutsch1996gzip, collet2016zstandard}.

Os fatores motivadores para a escolha deste problema incluem a crescente demanda por soluções eficientes de compressão de dados na era digital \cite{alakuijala2016brotli}, a necessidade de balancear taxa de compressão e tempo de execução em diferentes contextos e a relevância contínua dos algoritmos clássicos, mesmo diante do surgimento de novas técnicas \cite{collet2016zstandard}. Dessa forma, o estudo não apenas revisita fundamentos essenciais da compressão de dados, mas também propõe uma análise comparativa aplicada, fornecendo contribuições práticas e teóricas para a área.

