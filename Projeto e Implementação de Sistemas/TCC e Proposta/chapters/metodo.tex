\chapter{Metodologia}
\label{c.metodologia}

O estudo tem caráter \textbf{exploratório e descritivo}, pois busca compreender e comparar o desempenho de algoritmos clássicos de compressão de dados em diferentes tipos de arquivos. Além disso, trata-se de uma pesquisa \textbf{experimental e bibliográfica}, pois envolve tanto a revisão da literatura sobre os algoritmos estudados quanto a implementação prática e execução de testes para coleta de dados.

\section{Delineamento da pesquisa}

A pesquisa será conduzida em duas etapas principais:

\begin{itemize}
    \item \textbf{Levantamento de literatura}: estudo de referências acadêmicas sobre os algoritmos Huffman, LZ77, LZW e GZIP, incluindo seus princípios teóricos, implementação e aplicações.
    \item \textbf{Análise experimental}: implementação dos algoritmos e realização de testes comparativos, medindo a taxa de compressão e o tempo de execução em de arquivos de texto, áudio e imagem.
\end{itemize}

Com esse delineamento metodológico, espera-se obter uma avaliação quantitativa e qualitativa dos algoritmos estudados, possibilitando uma conclusão fundamentada sobre suas aplicações ideais.

\section{Levantamento de literatura}
O material bibliográfico será composto por livros, artigos científicos, teses, dissertações e fontes digitais relevantes para os algoritmos de compressão de dados. Serão consultados materiais teóricos que abordam os princípios, implementações e aplicações dos algoritmos Huffman, LZ77, LZW e GZIP. Além disso, serão utilizadas fontes que detalham a análise e comparação de desempenho desses algoritmos.

\section{Análise experimental}
Os algoritmos serão implementados nas linguagens de programação \texttt{C} ou \texttt{C++}. 

\subsection{Coleta de dados}
Os dados serão coletados por meio da execução dos algoritmos em um ambiente controlado, utilizando conjuntos de arquivos representativos dos três principais tipos de dados abordados: 

\begin{itemize}
    \item \textbf{Texto}: arquivos de livros e artigos científicos em formato \texttt{.txt}.
    \item \textbf{Imagem}: arquivos \texttt{.bmp}, que utilizam compressão sem perdas.
    \item \textbf{Áudio}: pequenos arquivos de áudio sem perdas em formato \texttt{.wav}.
\end{itemize}

Cada algoritmo será executado sobre os mesmos conjuntos de dados e os seguintes parâmetros serão registrados:

\begin{itemize}
    \item Taxa de compressão (\% de redução do tamanho do arquivo);
    \item Tempo de execução (medido em milissegundos);
\end{itemize}

\subsection{Análise dos dados}

Os resultados serão tabulados e analisados estatisticamente para identificar padrões e diferenças entre os algoritmos. A análise será feita considerando:

\begin{itemize}
    \item Comparação direta entre a taxa de compressão e tempo de execução;
    \item Eficiência relativa dos algoritmos para cada tipo de dado;
    \item Gráficos de desempenho desenvolvidos utilizando a linguagem de programação \texttt{Python} com as bibliotecas \texttt{pandas} e \texttt{matplotlib} a fim de facilitar a interpretação. Para gerar os gráficos coletaremos os dados e os colocaremos em um arquivo \texttt{.csv}. 
\end{itemize}

\subsection*{Ambiente controlado}

Os experimentos serão conduzidos em um ambiente computacional padronizado para garantir reprodutibilidade. 

\begin{itemize}
    \item Sistema operacional: Windows 11 Pro com Debian WSL;
    \item Hardware: Processador	AMD Ryzen 3 PRO 8300GE w/ Radeon 740M Graphics, 3501 Mhz, 4 Núcleo(s), 8 Processador(es) Lógico(s), 16GB RAM.
\end{itemize}




