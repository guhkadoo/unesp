\chapter{Objetivos}
\label{c.objetivos}

\section{Objetivo Geral}

Analisar comparativamente a eficiência dos algoritmos clássicos de compressão de dados (Huffman, LZ77, LZW e GZIP), avaliando suas taxas de compressão e tempos de execução em diferentes tipos de arquivos, a fim de determinar o impacto dessas variáveis na escolha do algoritmo mais adequado para aplicações práticas.

\section{Objetivos Específicos}

\begin{alineas}
	\item Descrever os princípios teóricos dos algoritmos de compressão de dados Huffman, LZ77, LZW e GZIP, destacando seus fundamentos e aplicações.
	\item Implementar os algoritmos selecionados, garantindo sua funcionalidade para compressão e descompressão de diferentes tipos de arquivos.
	\item Comparar o desempenho dos algoritmos em termos de taxa de compressão e tempo de execução, utilizando arquivos
    de texto TXT, imagens PNG e pequenos arquivos de áudio WAV.
	\item Analisar os resultados obtidos, identificando vantagens e limitações de cada algoritmo em diferentes cenários.
	\item Avaliar a aplicabilidade dos algoritmos em diferentes contextos, fornecendo recomendações para a escolha do método mais adequado conforme a necessidade específica.
\end{alineas}
