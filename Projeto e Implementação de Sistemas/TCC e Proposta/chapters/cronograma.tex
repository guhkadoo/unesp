\chapter{Cronograma}
\label{c.cronograma}

O cronograma a seguir apresenta a distribuição das atividades do Trabalho de Conclusão de Curso ao longo do período disponível, considerando os prazos estipulados pela instituição.

\begin{table}[htb]
    \centering
    \renewcommand{\arraystretch}{1.3}
    \begin{tabular}{|p{4cm}|p{5cm}|p{5cm}|}
        \hline
        \textbf{Semana} & \textbf{Atividades presenciais} & \textbf{Atividades extraclasse} \\
        \hline
        05 (24 a 30/03) & Redação da proposta do TCC & Desenvolvimento de biblioteca wav para lidar com .wav \\
        \hline
        06 (31/03 a 05/04) & Entrega da proposta do TCC & Desenvolvimento de Huffman Coding para .txt \\
        \hline
        07 (07 a 12/04) & Impressão da proposta e apresentação para os colegas & Desenvolvimento de Huffman Coding para .wav \\
        \hline
        08 (14 a 17/04) & Aguardando feedback da proposta & Continuação do desenvolvimento de Huffman Coding para .wav \\
        \hline
        09 (21 a 26/04) & Revisão e adequação da proposta (se necessário) & Desenvolvimento de Huffman Coding para .bmp \\
        \hline
        10 (28 a 30/04) & Postagem e reapresentação da proposta com adequações (se necessário) & Desenvolvimento de
        Huffman Coding para .bmp \\
        \hline
        11 a 15 (05/05 a 07/06) & - & Pesquisa bibliográfica aprofundada, leitura de artigos e livros, planejamento da
        implementação. Desenvolvimento do algoritmo LZ77 para .txt, .bmp e .wav \\
        \hline
        16 (09 a 14/06) & Preparação da apresentação do primeiro semestre, incluindo resumo do projeto e progresso até o momento & Desenvolvimento do algoritmo LZW para .txt \\
        \hline
        17 (16 a 18/06) & Apresentação do progresso do primeiro semestre (caso necessário) & Desenvolvimento do algoritmo LZW para .wav \\
        \hline
        \textbf{Recesso (19/06 a 03/08)} & - & Desenvolvimento da implementação dos algoritmos e testes preliminares
        (Desenvolvimento do algoritmo LZW para .bmp e do algoritmo Gzip para .txt, .bmp e .wav)\\
        \hline
        18 e 19 (04 a 16/08) & - & Ajustes nos algoritmos \\
        \hline
        20 (18 a 23/08) & Aula sobre regulamento do TCC e normas da ABNT & Testes de desempenho dos algoritmos, coleta de dados para análise comparativa\\
        \hline
        21 (25 a 30/08) & Apresentação do trabalho desenvolvido entre junho e agosto & Criação de gráficos com bibliotecas \texttt{Python} \\
        \hline
    \end{tabular}
    \label{tab:cronograma}
\end{table}

\begin{table}[htb]
    \centering
    \renewcommand{\arraystretch}{1.3}
    \begin{tabular}{|p{4cm}|p{5cm}|p{5cm}|}
	\hline
        \textbf{Semana} & \textbf{Atividades presenciais} & \textbf{Atividades extraclasse} \\
        \hline
        22 a 24 (01 a 20/09) & - & Criação de gráficos com bibliotecas \texttt{Python} \\
        \hline
        25 (22 a 27/09) & Apresentação das ferramentas utilizadas no desenvolvimento do TCC & - \\
        \hline
        26 (29/09 a 04/10) & Apresentação complementar das ferramentas (se necessário) & - \\
        \hline
        27 a 29 (06 a 25/10) & Entrega da cópia da monografia para cada membro da banca até 20/10 & Discussão dos resultados, revisão e finalização da monografia \\
        \hline
        30 (29/10 a 01/11) & - & Revisão e finalização da apresentação \\
        \hline
        31 (03/11 a 08/11) & - & Revisão da monografia \\
        \hline
        32 (10/11 a 14/11) & Realização das Bancas Examinadoras & - \\
        \hline
    \end{tabular}
    \caption{Cronograma de desenvolvimento do TCC}
\end{table}
