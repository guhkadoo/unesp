\section{Metodologia}
Os experimentos foram realizados por meio da implementação dos algoritmos Bubble Sort, Insertion Sort, Merge Sort, Quick
Sort, Heap Sort e Radix Sort em linguagem C. Para cada algoritmo, foram gerados vetores de diferentes tamanhos (1.000, 7.600, 14.200, 20.800, 27.400, 34.000, 40.600, 47.200, 53.800, 60.400, 67.000, 73.600, 80.200, 86.800, 93.400 e 100.000 elementos), organizados em três tipos de ordenação:
\begin{itemize}
    \item Crescente: vetor previamente ordenado em ordem crescente;
    \item Decrescente: vetor ordenado em ordem decrescente;
    \item Aleatória: vetor com elementos gerados de forma randômica.
\end{itemize}
Cada experimento foi executado cinco vezes, sendo registrado o tempo de execução médio, em milissegundos, para garantir maior confiabilidade nos resultados.
Os tempos de execução foram obtidos por meio da biblioteca time.h da linguagem de programação C. Os dados coletados
foram organizados em uma tabela csv e representados graficamente para facilitar a análise comparativa.

\begin{figure}[H]
\centering
\begin{forest}
for tree={
    font=\ttfamily,
    grow'=0,
    child anchor=west,
    parent anchor=east,
    anchor=west,
    calign=first,
    edge={},
    inner sep=4pt,
    l sep=2cm,
    before typesetting nodes={
      for tree={
        inner sep=0pt,
        outer sep=0pt,
        if level=0{font=\ttfamily\bfseries}{font=\ttfamily},
      }
    }
}
[codigos/
    [main.c]
    [sort.c                 \textit{<-- todas as funções de ordenação}]
    [sort.h]
    [scripts/
        [iterar.sh]
        [gerar\_grafico.py]
    ]
    [resultados/
        [resultados.csv]
        [bubble\_grafico.png]
        [insertion\_grafico.png]
        [\dots]
    ]
]
\end{forest}
\caption{Estrutura de diretórios do projeto.}
\label{fig:estrutura-diretorios}
\end{figure}

A estrutura do projeto foi organizada de forma modular, conforme ilustrado na Figura~\ref{fig:estrutura-diretorios}, com o objetivo de facilitar a manutenção, testes e análise de desempenho dos algoritmos de ordenação. A seguir, são descritas as funcionalidades dos principais componentes do projeto:

\begin{itemize}
    \item \texttt{main.c}: Arquivo principal que interpreta os argumentos passados via linha de comando, gera os vetores de entrada conforme o tipo solicitado (crescente, decrescente ou aleatório) e executa o algoritmo de ordenação correspondente.

    \item \texttt{sort.c} e \texttt{sort.h}: Contêm, respectivamente, a implementação e os protótipos das funções dos algoritmos de ordenação utilizados no projeto: Bubble Sort, Insertion Sort, Merge Sort, Quick Sort, Heap Sort e Radix Sort. O algoritmo a ser utilizado é selecionado de acordo com o parâmetro passado na execução.

    \item \texttt{scripts/}: Diretório com scripts auxiliares para automação dos testes e geração de gráficos:
    \begin{itemize}
        \item \texttt{iterar.sh}: Script em Bash responsável por automatizar a execução de testes para diferentes tamanhos de vetores, tipos de ordenação e algoritmos. Os resultados são armazenados em um arquivo \texttt{.csv}.
        \item \texttt{gerar\_grafico.py}: Script em Python que utiliza as bibliotecas \texttt{seaborn, numpy, pandas e
            matplotlib} para gerar gráficos de desempenho com base nos resultados obtidos.
    \end{itemize}

    \item \texttt{resultados/}: Diretório onde são armazenados os resultados dos testes e os gráficos gerados:
    \begin{itemize}
        \item \texttt{resultados.csv}: Arquivo contendo os dados brutos dos testes, incluindo o algoritmo utilizado, o tamanho do vetor, o tipo de ordenação e o tempo de execução.
        \item Arquivos \texttt{.png}: Gráficos de desempenho para cada algoritmo, gerados automaticamente a partir do script Python.
    \end{itemize}
\end{itemize}

Essa organização modular permitiu realizar testes sistemáticos e automatizados, além de facilitar a coleta e visualização dos dados, contribuindo para uma análise comparativa eficiente do desempenho dos algoritmos de ordenação.

\subsection{Principais Trechos de Código}

Nesta seção, são apresentados os principais trechos de código que compõem a base da execução dos experimentos, com foco na automação e análise dos resultados.

\subsubsection{Função Principal (\texttt{main.c})}

A função \texttt{main} é responsável por interpretar os parâmetros de entrada, gerar o vetor com base no tipo escolhido e executar o algoritmo de ordenação correspondente.

\begin{lstlisting}[language=C, caption={Função principal com medição de tempo}, label={lst:main}]
#include <stdio.h>
#include <stdlib.h>
#include <string.h>
#include <time.h>
#include "sort.h"

void preencher_vetor(int *v, int n, int tipo) {
    if (tipo == 0)
        for (int i = 0; i < n; i++) v[i] = i;
    else if (tipo == 1)
        for (int i = 0; i < n; i++) v[i] = n - i;
    else {
        srand(42);
        for (int i = 0; i < n; i++) v[i] = rand()%100000;
    }
}

int main(int argc, char *argv[]) {
    if (argc != 4) {
        printf("Uso: %s <algoritmo> <tamanho> <tipo>\n", argv[0]);
        printf("Algoritmos: bubble, insertion, merge, quick, heap, radix\n");
        return 1;
    }

    char *alg = argv[1];
    int n = atoi(argv[2]);
    int tipo = atoi(argv[3]);

    int *v = malloc(n * sizeof(int));
    preencher_vetor(v, n, tipo);

    clock_t ini = clock();

    if (strcmp(alg, "bubble") == 0)
        bubble_sort(v, n);
    else if (strcmp(alg, "insertion") == 0)
        insertion_sort(v, n);
    else if (strcmp(alg, "merge") == 0)
        merge_sort(v, 0, n-1);
    else if (strcmp(alg, "quick") == 0)
        quick_sort(v, 0, n-1);
    else if (strcmp(alg, "heap") == 0)
        heap_sort(v, n);
    else if (strcmp(alg, "radix") == 0)
        radix_sort(v, n);
    else {
        fprintf(stderr, "Algoritmo inválido: %s\n", alg);
        free(v);
        return 1;
    }

    clock_t fim = clock();
    double tempo = (double)(fim - ini) / CLOCKS_PER_SEC;

    printf("%s,%d,%d,%.6f\n", alg, n, tipo, tempo);

    free(v);
    return 0;
}
\end{lstlisting}

\subsubsection{Script de Execução Automática (\texttt{iterar.sh})}

Este script em Bash automatiza a execução dos testes, iterando sobre todos os algoritmos, tamanhos de vetor e tipos de ordenação. Os resultados são salvos em formato CSV para posterior análise.

\begin{lstlisting}[language=bash, caption={Script para automação dos testes}, label={lst:script}]
#!/bin/bash

ALGOS=("bubble" "insertion" "merge" "quick" "heap" "radix")
TAMANHOS=(1000 7600 14200 20800 27400 34000 40600 47200 53800 60400 67000 73600 80200 86800 93400 100000)
TIPOS=(0 1 2) # 0=crescente, 1=decrescente, 2=aleatório

mkdir -p ../resultados
echo "algoritmo,tamanho,tipo,tempo" > ../resultados/resultados.csv

gcc -O2 -o ../sorter ../main.c ../sort.c

for algo in "${ALGOS[@]}"; do
    echo "Executando testes para $algo..."
    for tipo in "${TIPOS[@]}"; do
        for n in "${TAMANHOS[@]}"; do
            soma=0
            for i in {1..5}; do
                resultado=$(../sorter $algo $n $tipo)
                tempo=$(echo "$resultado" | awk -F',' '{print $4}')
                soma=$(echo "$soma + $tempo" | bc -l)
            done
            media=$(echo "$soma / 5" | bc -l)
            echo "$algo,$n,$tipo,$media" >> ../resultados/resultados.csv
        done
    done
done    
\end{lstlisting}

\subsubsection{Script de Geração de Gráficos (\texttt{gerar\_grafico.py})}

O script em Python utiliza a biblioteca \texttt{pandas} para ler os resultados e \texttt{matplotlib/seaborn} para criar gráficos que comparam o desempenho dos algoritmos em diferentes cenários.

\begin{lstlisting}[language=Python, caption={Geração de gráficos com base nos dados}, label={lst:plot}]
import pandas as pd
import matplotlib.pyplot as plt
import seaborn as sns
import numpy as np

# Carregar os dados
df = pd.read_csv("../resultados/resultados.csv")

# Mapear tipos para nomes legíveis
mapa_tipos = {0: "Crescente", 1: "Decrescente", 2: "Aleatório"}
df["tipo"] = df["tipo"].map(mapa_tipos)

# Estilo bonito com seaborn
sns.set(style="whitegrid")

# Lista de algoritmos únicos
algoritmos = df["algoritmo"].unique()
print(algoritmos)

# Gerar um gráfico para cada algoritmo
for alg in algoritmos:
    dados_alg = df[df["algoritmo"] == alg]

    plt.figure(figsize=(8,5))

    # Plotar curvas para cada tipo de vetor
    for tipo in dados_alg["tipo"].unique():
        dados_tipo = dados_alg[dados_alg["tipo"] == tipo]
        plt.plot(dados_tipo["tamanho"], dados_tipo["tempo"], marker="o", label=tipo)

    plt.title(f"Desempenho do {alg.capitalize()} Sort")
    plt.xlabel("Tamanho do Vetor")
    plt.ylabel("Tempo (s)")
    plt.legend()
    plt.tight_layout()
    plt.savefig(f"../resultados/{alg}_grafico.png")
    plt.close()  # Fechar para não sobrepor gráficos
\end{lstlisting}

Esses trechos representam o núcleo da automação do projeto, permitindo a execução sistemática dos algoritmos e a análise gráfica dos seus desempenhos.

